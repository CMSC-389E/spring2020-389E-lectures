\documentclass{beamer}
\usetheme{AnnArbor}
\usecolortheme{spruce}
\usepackage{circuitikz}
\usepackage{graphicx}

\title{Introduction To Memory (ROM)}
\subtitle{Read-Only Memery}
\author[CMSC389E]{Akilesh Praveen | CMSC398E}
\institute{UMD}
\date{\today}



\begin{document}

    % title page
    \begin{frame}
        \titlepage
    \end{frame}
    
    % table of contents
    \begin{frame}
        \frametitle{Agenda}
        \tableofcontents
    \end{frame}
    
    \section{Announcements}
    
        \begin{frame}
                \vfill
                \centering
                \begin{beamercolorbox}[sep=8pt,center,shadow=true,rounded=true]{title}
                    \usebeamerfont{title}Announcements\par%
                \end{beamercolorbox}
                \vfill
             \end{frame}
    
        \subsection{Project 4}
        
            
            
            \begin{frame}
                \frametitle{Class cancelled next week!}
                \begin{itemize}
                    \item We won't be having class next week :)
                    \item Go enjoy your spring break!!
                    
                \end{itemize}
            \end{frame}
            
            \begin{frame}
            	\frametitle{Extension Policy}
            	\begin{itemize}
            		\item It's fine to ask for extensions, but please do so reasonably and \textbf{beforehand}.
            		\item We're already pretty lenient with grading in this class, but we will draw a line somewhere.
            		\item Note: If you have a medical note or a university excusal, this policy can be overriden.
            	\end{itemize}
            \end{frame}
            
    \section{Introduction to Memory}
    
    	\subsection{Introduction}
    	
    		\begin{frame}
                \vfill
                \centering
                \begin{beamercolorbox}[sep=8pt,center,shadow=true,rounded=true]{title}
                    \usebeamerfont{title}Read Only Memory\par%
                \end{beamercolorbox}
                \vfill
             \end{frame}
             
             \begin{frame}
             	\frametitle{What is Read Only Memory?}
             	\begin{itemize}
             		\item It's a pretty simple type of memory to understand, so we'll start off with it
             		\item Memory that you can write \textbf{once}, but you can only read from after
             		\item When you power off the machine, the memory you wrote will still remain the way you set it
             		
             	\end{itemize}
             \end{frame}
             
             \begin{frame}
             	\frametitle{Why Read Only Memory?}
             	\begin{itemize}
             		\item ROM has a lot of uses in modern electronics
             		\begin{itemize}
             			\item Things like BIOS in computers + other startup functions
             			\item Calculators for startup routines + repeated values
             			\item Put to heavy use in gaming consoles
             			\item Things like digital clocks and hair dryers also will have a fair bit of this stuff if you take them apart
             		\end{itemize}
             		
             	\end{itemize}
             \end{frame}
             
             \begin{frame}
             	\frametitle{Why Read Only Memory?}
             	\begin{itemize}
             		\item Incidentally, this is also the easiest memory to build 
             		\item We get the concept- and it turns out, there are easy ways to represent ROM as a set of functions
             	\end{itemize}
             \end{frame}
             
             \begin{frame}
             	\frametitle{Some types of ROM}
             	\begin{itemize}
             		\item \textbf{ROM} $\rightarrow$ Read Only Memory
             		\begin{itemize}
             			\item Data assigned during the manufacturing process
             		\end{itemize}
             		\item \textbf{PROM} $\rightarrow$ Programmable Read Only Memory
             	\end{itemize}
             \end{frame}
             
             
             
   
   		
    
\end{document}
